\documentclass[a4paper,10pt]{article}

\usepackage{graphicx}
\usepackage{titling}
\usepackage{enumerate}
\usepackage{listings}
\usepackage[table]{xcolor}
\newcolumntype{R}[1]{>{\raggedleft\arraybackslash }b{#1}}
\newcolumntype{L}[1]{>{\raggedright\arraybackslash }b{#1}}
\newcolumntype{C}[1]{>{\centering\arraybackslash }b{#1}}
\usepackage{color}
\usepackage{multirow}
\definecolor{mygreen}{rgb}{0,0.6,0}
\definecolor{mygray}{rgb}{0.5,0.5,0.5}
\definecolor{mymauve}{rgb}{0.58,0,0.82}
\lstset{%
	basicstyle=\scriptsize\sffamily,%
	commentstyle=\footnotesize\ttfamily,%
	frameround=trBL,
	frame=single,
	breaklines=true,
	showstringspaces=false,
	numbers=left,
	numberstyle=\tiny,
	numbersep=10pt,
	keywordstyle=\bf
	backgroundcolor=\color{white},   % choose the background color; you must add \usepackage{color} or \usepackage{xcolor}
  basicstyle=\footnotesize,        % the size of the fonts that are used for the code
  breakatwhitespace=false,         % sets if automatic breaks should only happen at whitespace
  breaklines=true,                 % sets automatic line breaking
  captionpos=b,                    % sets the caption-position to bottom
  commentstyle=\color{mygreen},    % comment style
  deletekeywords={...},            % if you want to delete keywords from the given language
  escapeinside={\%*}{*)},          % if you want to add LaTeX within your code
  extendedchars=true,              % lets you use non-ASCII characters; for 8-bits encodings only, does not work with UTF-8
  frame=single,                    % adds a frame around the code
  keepspaces=true,                 % keeps spaces in text, useful for keeping indentation of code (possibly needs columns=flexible)
  keywordstyle=\color{blue},       % keyword style
  language=Octave,                 % the language of the code
  morekeywords={*,...},            % if you want to add more keywords to the set
  numbers=left,                    % where to put the line-numbers; possible values are (none, left, right)
  numbersep=5pt,                   % how far the line-numbers are from the code
  numberstyle=\tiny\color{mygray}, % the style that is used for the line-numbers
  rulecolor=\color{black},         % if not set, the frame-color may be changed on line-breaks within not-black text (e.g. comments (green here))
  showspaces=false,                % show spaces everywhere adding particular underscores; it overrides 'showstringspaces'
  showstringspaces=false,          % underline spaces within strings only
  showtabs=false,                  % show tabs within strings adding particular underscores
  stepnumber=2,                    % the step between two line-numbers. If it's 1, each line will be numbered
  stringstyle=\color{mymauve},     % string literal style
  tabsize=2,                       % sets default tabsize to 2 spaces
  title=\lstname                   % show the filename of files included with \lstinputlisting; also try caption instead of title
}
\newcommand{\subtitle}[1]{%
	\posttitle{%
	\par\end{center}
\begin{center}\large#1\end{center}
\vskip0.5em}%
}


\title{Compilation Project}
\subtitle{Mid-Project Report\\ Master M1 MOSIG, Grenoble Universities}
\author{Cristian HARJA \and Yassine JAZOUANI \and Lina MARSS0 
\and Cl\'{e}ment MOMMESSIN}
\date{16/01/2015}

\begin{document}
% Beginning serious stuff. 


\maketitle

\section{Introduction}
\paragraph{}
Today, we are in the mid project. Therefore, we wrote this report to sum up all our work done. 
\paragraph{}
We begin by presenting you our planning of work for the last weeks and our logs,
then we list all the implemented functionalities, as well as the problems encountered with our solutions.
\paragraph{}
In the root folder of our project there is a "README.md" file, detailing the build instructions and structure of our project.
\section{Planning and log}
\paragraph{}
\begin{tabular}{|R{1.5cm}|C{7cm}|}
\hline \rowcolor{lightgray}Date & Implementation \\
\hline  12/01 & Intermediate Representation classes\\
\hline  13/01 & IR code generation from AST\\
\hline  14/01 & Finished code generation\\
\hline  15/01 & Assembly generation\\
\hline  16/01 & Working prototype, able to compile simple programs\\
\hline 
\end{tabular}

\section{Functionality implemented}
\begin{enumerate}
	\item Detection of free variables with unit tests
	\item Type Checking with unit tests
	\item Alpha Conversion with unit tests (for integers and float)
	\item Beta Reduction (for integers and float)
	\item Constant Folding
	\item Code Generation with an Intermediate Representation (IR)
	\item Assembly Code generation (integer computation, print\_newline and print\_int)
\end{enumerate}
\section{Problems encountered and solution found}
\begin{itemize}
	\item We had some minor problems concerning the IR. We couldn't find 
		  (in a timely fashion) a simple and efficient way of implementing it.
	\item We have some issues during the assembly code generation for float. We have some ideas 
		  but nothing concrete right now.
    \item We tried implementing stack management as well, but couldn't get it to work for this mid-project dead-line. For the moment we use global variables; this means recursive functions will not work properly.
    \item Some language constructs were difficult to compile (like closures), so for now we just stop the compilation with an error when encountering such a situation.
\end{itemize}
\section{Future steps}
\begin{itemize}
	\item Finish all other transformations on the AST (closure conversion, K-normalization, let-reduction...)
    \item Debug and improve our code generation (both IR and assembly).
    \item Implement optimizations on the IR.
    \item Follow the guidelines more closely, towards project completion.
\end{itemize}

\end{document}
